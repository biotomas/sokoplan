\documentclass[12pt, aspectratio=169]{beamer} % aspectratio = either 43 or 169

% colors (theme=...): red (default), blue, cyan, orange, green
%\usetheme[department=winuk,theme=cyan]{tue2008}
\usetheme{Warsaw}
\usecolortheme{beaver}
\setbeamercovered{transparent}
\mode<presentation>

%macros for sokoban level figures
\newcommand{\sokoimg}[1]{\includegraphics[scale=1.5]{#1} \hspace{-0.31em}}

\newcommand{\w}{\sokoimg{../figures/wall.pdf}}
\newcommand{\e}{\sokoimg{../figures/empty.pdf}}
\newcommand{\p}{\sokoimg{../figures/player.pdf}}
\newcommand{\x}{\sokoimg{../figures/box.pdf}}
\newcommand{\g}{\sokoimg{../figures/goal.pdf}}
\newcommand{\h}{\sokoimg{../figures/goalbox.pdf}}
\newcommand{\n}{\\
\vspace{-0.1em}}


% title and author
%
%\titlerunning{Abbreviated paper title}
% If the paper title is too long for the running head, you can set
% an abbreviated paper title here
%
%
% First names are abbreviated in the running head.
% If there are more than two authors, 'et al.' is used.
%

\title{AI Assisted Design of Sokoban Puzzles\\
using Automated Planning}
\author{Tom\'a\v{s} Balyo\inst{1} \and
Nils Froleyks\inst{2}}

%\institute{CAS Software AG, Karlsruhe, Germany\\
%\email{tomas.balyo@cas.de}
%\and
%Johannes Kepler University, Linz, Austria\\
%\email{nils.froleyks@jku.at}}



\institute[Universities Here and There] % (optional)
{
  \inst{1}%
  CAS Software, Karlsruhe, Germany\newline
  tomas.balyo@cas.de
  \and
  \inst{2}%
  Johannes Kepler University, Linz, Austria\newline
  nils.froleyks@jku.at
}
\date[KPT 2021] % (optional)
{EAI ArtsIT 2021 – 10th EAI International Conference: ArtsIT,\newline Interactivity \& Game Creation}
\subject{Computer Science}
% document
\begin{document}

\begin{frame}[plain]\titlepage\end{frame}

\begin{frame}{Contents of the Talk}
  \begin{itemize}
    \item What is Sokoban
    \item Designing levels for Sokoban
    \item How does our new level design tool work
    \item Usage demonstration
    \item Evaluation and Conclusion
  \end{itemize}
\end{frame}

\begin{frame}{What is Sokoban}
\begin{columns}[c] % the "c" option specifies center vertical alignment
    \column{.69\textwidth} % column designated by a command
  \begin{itemize}
    \item A puzzle game originally from Japan in 1982
    \pause
    \item A level (warehouse) consists of a single worker and multiple walls, boxes and goal positions
    \pause
    \item The player can move the worker in the 4 cardinal directions (up, down, left, right)
    \pause
    \item The worker cannot walk through walls
    \pause
    \item The worker can push a box if the tile behind it is empty (no wall or other box there)
    \pause
    \item The goal is to push the boxes onto goal positions
  \end{itemize}
    \column{.29\textwidth}
\only<1-2>{    
\begin{figure}
\centering
\w\w\w\w\e\n
\w\e\p\w\e\n
\w\x\w\w\w\n
\w\e\x\g\w\n
\w\g\w\w\w\n
\w\w\w\e\e\n
\end{figure}
}
\only<3-4>{    
\begin{figure}
\centering
\w\w\w\w\e\n
\w\p\e\w\e\n
\w\x\w\w\w\n
\w\e\x\g\w\n
\w\g\w\w\w\n
\w\w\w\e\e\n
\end{figure}
}
\only<5>{    
\begin{figure}
\centering
\w\w\w\w\e\n
\w\e\e\w\e\n
\w\p\w\w\w\n
\w\x\x\g\w\n
\w\g\w\w\w\n
\w\w\w\e\e\n
\end{figure}
}
\only<6>{    
\begin{figure}
\centering
\w\w\w\w\e\n
\w\e\e\w\e\n
\w\e\w\w\w\n
\w\e\p\h\w\n
\w\h\w\w\w\n
\w\w\w\e\e\n
\end{figure}
}
\end{columns}
\end{frame}

\begin{frame}{Why Sokoban?}
  \begin{itemize}
    \item Sokoban is well known in Video Games Communities
    \begin{itemize}
    	\item Multiple implementations on almost all existing platforms
    	\item Thousands of levels created by designers and fans
    \end{itemize}
    \item Sokoban is also well known in Computer Science Research
    \begin{itemize}
    	\item In Complexity Theory -- it was proven to be P-SPACE complete
    	\item In Artificial Intelligence -- numerous papers on solving and generating Sokoban levels have been published
    \end{itemize}
  \end{itemize}
\end{frame}

\begin{frame}{What is a good Sokoban Level?}
  \begin{itemize}
    \item A level should be solvable
    \begin{itemize}
    	\item unsolvable levels are interesting in research but frustrating for human players
    	\item easy to guarantee in automatic generation
    \end{itemize}
    \item A level should be challenging
    \begin{itemize}
    	\item avoid levels that are solved with very few steps
    	\item avoid levels with obvious solutions
    	\item challenging to ensure in automatic generation
    \end{itemize}
    \item A level should be visually attractive for human players
    \begin{itemize}
    	\item very difficult to achieve without a human designer
    \end{itemize}
    
  \end{itemize}
\end{frame}

\begin{frame}{How our Generator Works}
\begin{itemize}
\item basic idea -- translate the task of designing a level into a standard planning problem and use a state-of-the-art automated planner to solve it
\pause
\item what is a planning problem?
\begin{itemize}
\item given a set of possible actions, initial state and goal conditions
\item find a sequence of actions that can get us from the initial state to a state where all goal
conditions are met
\item a well researched problem with very efficient solver programs available
\end{itemize}
\end{itemize}
\end{frame}

\begin{frame}{How our Generator Works}
\begin{itemize}
\item how to express level design as a planning problem?
\pause
\begin{itemize}
\item the initial state is an unfinished level where additional boxes, walls, and the worker can
be added at certain locations -- specified by the human designer
\pause
\item the goal conditions are that the specified number of boxes, walls, and the worker is added and the level is solvable
\pause
\item the possible actions are placing items and playing the game
\pause
\item the final plan has two parts, the first part is placing items to finish the level, the second is solving the level
\end{itemize}
\end{itemize}
\end{frame}


\begin{frame}[fragile]{How to use our Generator}
There is a standard text format for Sokoban levels
\begin{columns}[c] % the "c" option specifies center vertical alignment
    \column{.39\textwidth} % column designated by a command
\begin{figure}
\centering
\w\w\w\w\e\n
\w\e\p\w\e\n
\w\x\w\w\w\n
\w\e\x\g\w\n
\w\g\w\w\w\n
\w\w\w\e\e\n
\end{figure}

    \column{.59\textwidth} % column designated by a command
\begin{verbatim}
####     symbols:   
# @#      # - wall          
#$###     @ - worker        
# $.#     + - worker on goal
#.###     $ - box        
###       . - goal        
          * - box on goal
          
\end{verbatim}
\end{columns}
\end{frame}

\begin{frame}[fragile]{How to use our Generator}
We extend the format with additional symbols to specify possible locations of additional
walls, boxes, and the worker.
\begin{columns}[c] % the "c" option specifies center vertical alignment
    \column{.29\textwidth} % column designated by a command
\begin{figure}
\centering
\w\w\w\w\e\n
\w\e\e\w\e\n
\w\e\w\w\w\n
\w\e\e\g\w\n
\w\g\w\w\w\n
\w\w\w\e\e\n
\end{figure}

    \column{.69\textwidth} % column designated by a command
\begin{verbatim}
####   symbols:            what can be added:
# 1#    # - wall            0 - anything
#0###   @ - worker          1 - worker  
#00.#   + - worker on goal  2 - wall    
#.###   $ - box             3 - box           
###     . - goal            4 - worker or wall
        * - box on goal     5 - worker or box
                            6 - wall or box
\end{verbatim}
\end{columns}
\end{frame}

\begin{frame}{How our Generator Works}
\begin{enumerate}
\item use any text editor to create level template as shown above
\item save to a text file, for example \textit{level.txt}
\item download our tool from \url{https://github.com/biotomas/sokoplan} and build it following the instructions
\item run the tool with the command \texttt{./run.sh level.txt}
\item the generated level will be saved to a file \textit{level.txt.solution}
\item you can see and test your level using the included \textit{levelTester.html}
\end{enumerate}
\end{frame}

\begin{frame}{Demonstration}
\end{frame}

\begin{frame}{Previous vs Our Generator}
\begin{itemize}
\item Previous work
\begin{itemize}
\item fully automatic level generators
\item based on the principle of generating plenty of levels and then filter the good ones
\item the search algorithm is fully integrated, the generator does not improve over time without extensive maintenance
\end{itemize}
\item Our Generator
\begin{itemize}
\item the tool utilizes human input
\item based on the principle of searching for the good level directly
\item the search is outsourced to Automated Planners, which get better over time improving the generator tool automatically
\end{itemize}
\end{itemize}
\end{frame}

\begin{frame}{Conclusion}
\begin{itemize}
\item We developed a new tool to assist a human level designer in developing Sokoban puzzle levels. 
\item It has some limitations
\begin{itemize}
\item the outline of the level must be provided
\item all the goals must be placed by the designer
\item performance problems with large levels
\item very technical user interface
\end{itemize}
\item Future work
\begin{itemize}
\item fix the limitations
\item use the approach to develop tools for other similar games
\end{itemize}

\end{itemize}
\end{frame}

\end{document}