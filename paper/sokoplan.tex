% This is samplepaper.tex, a sample chapter demonstrating the
% LLNCS macro package for Springer Computer Science proceedings;
% Version 2.20 of 2017/10/04
%
\documentclass[runningheads]{llncs}
%
\usepackage{graphicx}
% Used for displaying a sample figure. If possible, figure files should
% be included in EPS format.
%
% If you use the hyperref package, please uncomment the following line
% to display URLs in blue roman font according to Springer's eBook style:
% \renewcommand\UrlFont{\color{blue}\rmfamily}

\begin{document}
%
\title{AI Guided Design of Sokoban Puzzles based on Automated Planning}
%
%\titlerunning{Abbreviated paper title}
% If the paper title is too long for the running head, you can set
% an abbreviated paper title here
%
\author{Tom\'a\v{s} Balyo\inst{1} \and
Nils Froleyks\inst{2}}
%
\authorrunning{Tom\'a\v{s} Balyo, Nils Froleyks}
% First names are abbreviated in the running head.
% If there are more than two authors, 'et al.' is used.
%
\institute{CAS Software AG, Karlsruhe, Germany\\
\email{tomas.balyo@cas.de}
\and
Johannes Kepler University, Linz, Austria\\
\email{nils.froleyks@jku.at}}
%
\maketitle              % typeset the header of the contribution
%
\begin{abstract}
Designing interesting and challenging levels for a puzzle game is a very difficult and time
consuming task. It is often possible to develop random puzzle generators that can produce
solvable levels. However, in order to obtain appealing levels, usually a human designer 
needs to be involved. In this paper we propose a new generic method for assisting human
designers to create solvable levels for a puzzle game by using Automated Planning. 
We will demonstrate our method on the well-known Japanese puzzle game Sokoban.

\keywords{First keyword  \and Second keyword \and Another keyword.}
\end{abstract}
%
%
%
\section{Introduction}

\section{Preliminaries}
TODO definition of sokoban

TODO definition of automated planning

\section{Related Work}
TODO other sokoban level generators

\section{Puzzle Generation as Planning}

\subsection{Sokoban Solving as Planning}

\subsection{Level Creation as Planning}

\subsection{Dealing with Trivial Levels}
The planner tries to find short plans, it will try to place boxes next to goals

solution: minimal number of pushes

\subsubsection{Dealing with Cyclic Pushes}

\section{Experimental Evaluation}
how does it scale?

what is the largest level we can create?

\section{Conclusion}
\subsection{Future Work}

%
% ---- Bibliography ----
%
% BibTeX users should specify bibliography style 'splncs04'.
% References will then be sorted and formatted in the correct style.
%
\bibliographystyle{splncs04}
% \bibliography{mybibliography}
%
\end{document}
