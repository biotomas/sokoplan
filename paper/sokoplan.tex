% This is samplepaper.tex, a sample chapter demonstrating the
% LLNCS macro package for Springer Computer Science proceedings;
% Version 2.20 of 2017/10/04
%
\documentclass[runningheads]{llncs}
%
\usepackage{graphicx}
% Used for displaying a sample figure. If possible, figure files should
% be included in EPS format.
%
% If you use the hyperref package, please uncomment the following line
% to display URLs in blue roman font according to Springer's eBook style:
% \renewcommand\UrlFont{\color{blue}\rmfamily}

\begin{document}
%
\title{AI Guided Design of Sokoban Puzzles based on Automated Planning}
%
%\titlerunning{Abbreviated paper title}
% If the paper title is too long for the running head, you can set
% an abbreviated paper title here
%
\author{Tom\'a\v{s} Balyo\inst{1} \and
Nils Froleyks\inst{2}}
%
\authorrunning{Tom\'a\v{s} Balyo, Nils Froleyks}
% First names are abbreviated in the running head.
% If there are more than two authors, 'et al.' is used.
%
\institute{CAS Software AG, Karlsruhe, Germany\\
\email{tomas.balyo@cas.de}
\and
Johannes Kepler University, Linz, Austria\\
\email{nils.froleyks@jku.at}}
%
\maketitle              % typeset the header of the contribution

%macros for sokoban level figures
\newcommand{\sokoimg}[1]{\includegraphics[scale=2]{#1} \hspace{-0.35em}}

\newcommand{\w}{\sokoimg{figures/wall.pdf}}
\newcommand{\e}{\sokoimg{figures/empty.pdf}}
\newcommand{\p}{\sokoimg{figures/player.pdf}}
\newcommand{\x}{\sokoimg{figures/box.pdf}}
\newcommand{\g}{\sokoimg{figures/goal.pdf}}
\newcommand{\h}{\sokoimg{figures/goalbox.pdf}}
\newcommand{\n}{\\
\vspace{-0.09em}}

%
\begin{abstract}
Designing interesting and challenging levels for a puzzle game is a very difficult and time
consuming task. It is often possible to develop random puzzle generators that can produce
solvable levels. However, in order to obtain appealing levels, usually a human designer 
needs to be involved. In this paper we propose a new generic method for assisting human
designers to create solvable levels for a puzzle game by using Automated Planning. 
We will demonstrate our method on the well-known Japanese puzzle game Sokoban.

\keywords{First keyword  \and Second keyword \and Another keyword.}
\end{abstract}
%
%
%
\section{Introduction}

TODO 6-11 pages

Sokoban is a puzzle game that originated in Japan. It was invented by Hiroyuki
Imabayashi, and published in 1982 by Thinking Rabbit, as a PC game.\footnote{
  \url{https://en.wikipedia.org/wiki/Sokoban} (10.11.2016) } The word Sokoban is Japanese for warehouse keeper.

Each level represents a warehouse, where boxes are randomly placed. A warehouse
keeper has to push the boxes around the warehouse so that all boxes are on goals
at the end of the game.

The game of Sokoban is a complicated computational problem. It was first proven
to be NP-hard \cite{dorit96} and then PSPACE-complete \cite{culberson97}. While
the rules are simple, even small levels can require a lot of computation to be
solved.


\begin{figure}
\centering
\begin{verbatim}
(:action place_wall
 :parameters(?w - wall ?to - tile)
 :precondition(and
   (making_level)
   (opt_wall ?to)
   (not (wall_placed ?w))
   (not (wall ?to))
   (not (box ?to))
 )
 :effect(and
   (wall ?to)
   (wall_placed ?w)
 )
)

\end{verbatim}
\caption{Sample PDDL Code}
\end{figure}

Our goal is to generate Sokoban levels which fulfill the following three condition
specified by Murase et.al~\cite{murase1996automatic}:
\begin{enumerate}
\item Generate original problems
\item Generate problems that have solutions
\item Generate interesting (non-trivial) problems
\end{enumerate}

\section{Preliminaries}

\begin{figure}
\centering
\w\e\x\e\g\e\p\n
\caption{The four kinds of tiles that make up a Sokoban warehouse: Wall, Box, Goal, and Player (from left to right).}.
\label{fig-tiles}
\end{figure}


\begin{figure}
\centering
\w\w\w\w\w\w\e\w\w\w\w\w\w\n
\w\p\e\x\g\w\e\w\e\e\p\h\w\n
\w\w\w\w\w\w\e\w\w\w\w\w\w\n
\caption{A simple Sokoban level in its initial (left) and solved (right) state. 
The solution to this level consists of two steps: MOVE-RIGHT and PUSH-RIGHT.}.
\label{fig-basic-level}
\end{figure}


Each Sokoban level consists of a two dimensional rectangular grid of squares that make
up the "warehouse" (See Figure~\ref{fig-basic-level} for an example). The squares are indexed starting from the top left with
$(0,0)$.
If a square contains nothing it is called a floor. Otherwise it is occupied by
one of the following entities:

\begin{enumerate}
\item \emph{Wall.} Walls make up the basic outline of each level. They cannot be
  moved and nothing else can be on a square occupied by a wall. A legal level is
  always surrounded by walls.

\item \emph{Box.} A box can either occupy a goal or an otherwise empty square. They can be
  moved in the four cardinal directions by \emph{pushing}.

  A \emph{push} is defined by a \emph{start position} and a \emph{direction}.
  The start position is given by a tuple of indices $(x,y)$.

  The direction can be either $up$, $down$, $left$ or $right$.

  As figured below let us suppose the direction to be $right$.

  For a push to be legal the following conditions must be met:
  \begin{itemize}
  \item A box is at position $(x, y)$.
  \item The player can reach the position $(x-1, y)$.
  \item Position $(x + 1, y)$ is either floor or a goal.
  \end{itemize}
  After executing the push the box is located at position $(x + 1, y)$ and the
  player at position $(x,y)$.

%  \begin{figure}[h!]
%    \centering
%    \includegraphics[width=0.4\textwidth]{graphics/pushPre.png}
%    \includegraphics[width=0.4\textwidth]{graphics/pushPost.png}
%    \caption{A push to the right}
%  \end{figure}

\item  \emph{Goal.} Goals are treated like floors for the most part. Only when
  each goal is occupied by a box the game is completed. In a legal level the
  number of goals matches the number of boxes. For the sake of simplicity, we
  will call a square that is either a goal or a floor square \emph{free} since
  the player and boxes can enter both.

\item \emph{Player} The player can execute \emph{moves} to alter his position.

  A \emph{move} can be $up$, $down$, $left$ or $right$. A move is also a push if
  it alters the position of a box.

  The player cannot move through walls or boxes. It can only move onto a square
  occupied by a box if it can execute a push to move it out of the way.
  Therefore the player cannot push more than one box at a time, nor can he pull
  them.
% \end{itemize}
\end{enumerate}


The goal of the game is to find a \emph{solution}, which
is a sequence of moves and pushes. Executing a solution leads to every box
being on a goal. It does not matter which box ends up on which goal.


A level with solution is given in figure \ref{fig:short}.

%\begin{figure}[h!]
%  \centering
%  \includegraphics[width=0.45\textwidth]{graphics/shortSolution.png}
%  \caption{A possible solution string: $rddLruulDuullddR$}
%  \label{fig:short}
%\end{figure}


TODO definition of automated planning

\section{Related Work}
The first available Sokoban level generator is by Murase et al.~\cite{murase1996automatic}.
Their approach has three phases.
\begin{enumerate}
\item \emph{Generate random levels}. In this phase predefined templates of walls are placed randomly
over a prototype level consisting of only walls. 
The templates are placed such that they are connected by passages.
Then boxes and goal tiles are placed randomly.
\item \emph{Filter out unsolvable levels}. Phase one may generate levels that have no solution. According to the the authors this happens in around half of the cases. In this phase they use a sokoban solver to try to find a solution and filter out unsolvable levels.
\item \emph{Evaluation}. In this phase the levels are automatically evaluated to determine whether
they are interesting. The evaluation is based on simple metrics such as the length of the solution, the number of changes in directions when pushing a box and the number of detours.
\end{enumerate}

The complexity of this approach is dominated by phase two -- filtering out unsolvable levels. This
step requires solving Sokoban problems, which is a PSPACE-complete problem~\cite{culberson1997sokoban}.

The approach of Taylor and Parberry~\cite{taylor2011procedural} is similar to Murase et al.
in that they first generate a random level based on placing templates of walls. Then they
randomly place goals with boxes on them in the rooms. At this point they 
actually have a solved Sokoban
puzzle. In the following stage they ``unsolve'' the level by doing reverse Sokoban moves, i.e., 
pulling boxes away from the goals. The aim of this stage is to reach a state that is far as 
possible from the solved state. They do this by running an iterative deepening search of the
state space.

The complex part of this algorithm is the search for the starting state in the second stage. 
The process is very memory intensive, since all the visited states have to be kept in 
memory in order to avoid looping. On the other hand, the algorithm has the anytime property, i.e., it 
can be stopped at any time to return a valid solution, however, letting it run longer will yield
a better solution.

In \cite{taylor2015attention} an auditory Stroop test was performed to compare the engagement of
players while playing hand-crafted Sokoban levels against levels generated by the approach of 
Taylor and Parberry~\cite{taylor2011procedural}. The experiment showed that players found
procedurally generated levels equally interesting to hand-crafted levels. This demonstrates
that there is entertainment value in procedurally generated puzzles.

Kartal et al.~\cite{kartal2016data} propose a Monte Carlo tree search based Sokoban level
generator.
TODO more detailed description

An up-to-date survey on procedural puzzle generation~\cite{de2019procedural} gives an overview
of the methods for generating puzzles for Sokoban and many other similar games.

\section{Puzzle Generation as Planning}

\subsection{Sokoban Solving as Planning}

\subsection{Level Creation as Planning}

\subsection{Dealing with Trivial Levels}
The planner tries to find short plans, it will try to place boxes next to goals

solution: minimal number of pushes

\subsubsection{Dealing with Cyclic Pushes}

\section{Experimental Evaluation}
how does it scale?

what is the largest level we can create?

\section{Conclusion}
\subsection{Future Work}

%
% ---- Bibliography ----
%
% BibTeX users should specify bibliography style 'splncs04'.
% References will then be sorted and formatted in the correct style.
%
\bibliographystyle{splncs04}
\bibliography{references, literatur}
%
\end{document}
